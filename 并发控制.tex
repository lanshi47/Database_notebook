\section{并发控制}
\subsection{并发控制的基本概念}
并发控制:通过锁机制(悲观控制)或多版本控制(乐观控制)确保事务的一致性和隔离性。\\
封锁:在悲观控制中,事务对数据项加锁,防止其他事务同时访问导致数据不一致。\\
数据不一致性:
\begin{enumerate}
    \item 丢失更新
    \item 脏读
    \item 不可重复读
    \item 幻读
\end{enumerate}
\subsection{封锁的基本概念}
排他锁(X锁/写锁):事务加锁后禁止其他事务对该数据项进行读写。\\
共享锁(S锁/读锁):允许其他事务同时加共享锁读取,但禁止加排他锁进行写操作。\\

\subsection{封锁协议}
\begin{enumerate}
    \item 一级封锁协议:事务对数据项加锁后,直到事务结束才释放锁。
    \item 严格两段锁协议(Strict 2PL):事务在整个执行期间只在结束时统一释放所有锁,避免脏读问题。
    \item 两段锁协议(2PL):事务分为加锁阶段和解锁阶段,加锁阶段期间不释放锁,进入解锁阶段后不能再申请新锁。
\end{enumerate}
\subsection{活锁和死锁}
活锁:多个事务不断响应彼此的请求,导致无法有效推进。\\
死锁:多个事务形成相互等待关系,导致系统僵持。\\
死锁处理方法:
\begin{enumerate}
    \item 死锁检测:构建等待图,检测循环依赖。
    \item 死锁恢复:通过回滚部分事务解除死锁。
    \item 死锁预防:采用资源排序、一次性申请所有资源等策略。
\end{enumerate}
\subsection{可串行化调度}
可串行化调度:事务执行顺序经过调整后效果等同于某一串行顺序,保证数据的一致性。\\

\subsection{两段锁协议(2PL)}
两段锁协议:事务执行分为加锁阶段(不释放任何锁)和解锁阶段(统一释放所有锁)。\\

\subsection{多版本并发控制(MVCC)}
多版本并发控制:通过保存数据的多个版本,使得读操作无需等待写锁,从而提高并发性能。常见于乐观并发控制策略。\\

