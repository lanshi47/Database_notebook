

\section{第一章:绪论}

\subsection{数据库的4个基本概念}
\begin{enumerate}
    \item 数据data
    \item 数据库database,DB
    \item 数据库管理系统DBMS
    \item 数据库系统DBS
\end{enumerate}
\subsection{数据库系统的特点}
\begin{enumerate}
    \item 结构化
    \item 共享性高,低冗余,易扩充
    \item 数据独立性高:物理;逻辑
    \item 由DBMS统一管理和控制
\end{enumerate}
\subsection{数据模型}
\begin{enumerate}
    \item 概念模型-E-R图
    \item 逻辑模型--关系模型
    \item 物理模型
\end{enumerate}
\subsection{数据模型的组成要素:数据结构,数据操作,数据的完整性约束条件}
\begin{enumerate}
    \item 数据结构-静态
    \item 数据操作-动态
    \item 完整性约束条件
\end{enumerate}
\subsection{\color{red}\textbf{重点:数据库系统的三级模式结构:外模式,模式(逻辑模式),内模式}}
\begin{enumerate}
    \item 外模式:看到的,可视化的
    \item 模式:???
    \item 内模式:之间的关系
\end{enumerate}
\subsection{数据库的二级印象功能与逻辑独立性}
\begin{enumerate}
    \item 外模式/模式:保证了数据的逻独立性
    \item 模式/内模式:保证了~物理独立性
\end{enumerate}
