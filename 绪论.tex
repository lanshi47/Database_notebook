

\section{第一章:绪论}

\subsection{数据库的4个基本概念}
\begin{enumerate}
    \item 数据data
    \item 数据库database,DB
    \item 数据库管理系统DBMS
    \item 数据库系统DBS
\end{enumerate}
\subsection{数据库系统的特点}
\begin{enumerate}
    \item 结构化
    \item 共享性高,低冗余,易扩充
    \item 数据独立性高:物理;逻辑
    \item 由DBMS统一管理和控制
\end{enumerate}
\subsection{数据模型}
\begin{enumerate}
    \item 概念模型-E-R图
    \item 逻辑模型--关系模型
    \item 物理模型
\end{enumerate}
\subsection{数据模型的组成要素:数据结构,数据操作,数据的完整性约束条件}
\begin{enumerate}
    \item 数据结构-静态
    \item 数据操作-动态
    \item 完整性约束条件
\end{enumerate}
\subsection{\texorpdfstring{\color{red}\textbf{重点:数据库系统的三级模式结构:外模式,模式(逻辑模式),内模式}}{重点:数据库系统的三级模式结构:外模式,模式(逻辑模式),内模式}}
\begin{enumerate}
    \item 外模式是用户或应用程序看到的局部数据逻辑结构,也称为用户视图。
    每个外模式为特定用户组定制,屏蔽了数据库的复杂结构。
    例如,不同用户可能通过不同的外模式访问同一数据库的不同部分。
    \item 模式是数据库的全局逻辑结构,定义所有数据实体、属性、关系及约束(如主键、外键)。例如,包含所有表的结构及其联系,是数据库设计的核心蓝图。
    \item 内模式描述数据的物理存储方式,如文件组织、索引结构、数据压缩等。例如,决定数据以B+树索引存储,或使用堆文件方式。
\end{enumerate}
外模式是用户视角,模式是全局逻辑蓝图,内模式是物理存储细节。
\subsection{数据库的二级印象功能与逻辑独立性}
\begin{enumerate}
    \item 外模式/模式:保证了数据的逻独立性
    \item 模式/内模式:保证了~物理独立性
\end{enumerate}
