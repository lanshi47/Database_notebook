\documentclass[a4paper,12pt,UTF8,fontset=none]{ctexart}
\usepackage{geometry} % 页面设置
\usepackage{xeCJK}
\usepackage{enumitem}
\usepackage{xcolor} % Color support for listings
\usepackage{graphicx}
\usepackage{amssymb} % For join symbol
\usepackage{amsmath}
\usepackage{listings}
\usepackage{longtable}
\usepackage{booktabs} % 添加booktabs宏包以支持三线表
\usepackage{array}      % 列格式控制
\usepackage{caption}    % 表格标题格式
\usepackage{placeins} 
\usepackage{newtxtext,newtxmath}
\usepackage{fontspec}
\usepackage{titlesec}
\usepackage{cleveref}

% 设置英文主字体为 Times New Roman
\setmainfont{Times New Roman}[Path=D:/Program Files/MiKTeX/fonts/custom/, Extension=.otf]

% 设置英文粗体字体为 Times New Roman Bold
\newfontfamily\enboldfont{Times New Roman Bold}[Path=D:/Program Files/MiKTeX/fonts/custom/, Extension=.otf]

% 设置中文正文字体为宋体
\setCJKmainfont{SimSun}[Path=D:/Program Files/MiKTeX/fonts/custom/, Extension=.ttf]

% 设置中文黑体字体
\setCJKfamilyfont{zhhei}{SimHei}[Path=D:/Program Files/MiKTeX/fonts/custom/, Extension=.ttf]
\newcommand{\heiti}{\CJKfamily{zhhei}}


% 自定义编号格式
\renewcommand{\thesection}{\arabic{section}}
\renewcommand{\thesubsection}{\thesection.\arabic{subsection}}
\renewcommand{\thesubsubsection}{\thesubsection.\arabic{subsubsection}}
% 设置标题格式
\titleformat{\section}
  {\centering\heiti\zihao{-3}\enboldfont}{\thesection}{1em}{}
\titleformat{\subsection}
  {\heiti\zihao{4}\enboldfont}{\thesubsection}{1em}{}
\titleformat{\subsubsection}
  {\heiti\zihao{-4}\enboldfont}{\thesubsubsection}{1em}{}



% 设置代码块样式
\lstset{
    basicstyle=\ttfamily,
    columns=fullflexible,
    frame=single,
    breaklines=true,
    postbreak=\mbox{\textcolor{red}{$\hookrightarrow$}\space},
    language=SQL,
    keywordstyle=\color{blue},
    commentstyle=\color{gray},    
    rulecolor=\color{black!30},%边框颜色
    stringstyle=\color{red},
    escapeinside={\%*}{*)},
    showstringspaces=false,
    captionpos=b % 设置标题位置, b表示在底部
}
\geometry{left=2.5cm,right=2.5cm,top=2.5cm,bottom=2.5cm} % 页边距


\begin{document}

% 制作封面页
\begin{titlepage}
    \centering
    \vspace*{\fill}
    {\LARGE\bfseries 数据库系统原理课程\par}
    \vspace{2cm}
    {\Huge\bfseries 往年真题\par}
    \vspace{2cm}
    {\Large 烂石\par}
    \vspace{1cm}
    {\large \today \par}
    \vspace{4cm}
    \includegraphics[width=0.5\textwidth]{static/images/logo.jpg}
    \vspace*{\fill}
    \thispagestyle{empty} % 封面不显示页码
    \newpage
\end{titlepage}

% 正文内容
\section{选择题(40 points)}
\subsection{日志文件的作用}
日志文件是数据库系统中至关重要的组件,主要用于保障事务的ACID特性(原子性、一致性、隔离性、持久性) ,确保数据的一致性和系统故障后的恢复能力。
\subsection{为什么要应用数据库}
\begin{itemize}
    \item 共享与高效利用
    \item 减少冗余与提高一致性
    \item 支持并发与安全性
    \item 数据独立性与事务处理
\end{itemize}
\subsection{DB/DBS/DBMS之间关系}
\subsubsection{层级关系}
DBS(系统) > DBMS(软件) > DB(数据)

DB是数据存储,DBMS是管理DB的软件,DBS是整体系统。
\subsubsection{核心功能区分}
DBMS:提供数据管理功能(如事务、安全、查询优化)。
DBS:包含DBMS、DB、应用程序及用户,是完整的数据管理解决方案。
\subsubsection{常见误区}
混淆DB和DBMS(如认为“数据库就是MySQL”)。
认为DBS仅包含DB和DBMS,忽略应用程序和用户。
\section{填空题 30 points}
\subsection{逻辑独立性是什么?}
逻辑独立性是指当数据的逻辑结构(概念模式)改变时,应用程序不需要修改。
\subsection{把一个sql语句转化成关系代数}
\subsubsection{填空标准答案示例:}
将SQL语句 `SELECT A, B FROM R WHERE C > 5` 转换为关系代数表达式:  \\
π<sub>A,B</sub>(σ<sub>C>5</sub>(R))\\
(其中:  
- π 表示投影(对应SELECT字段),  \\
- σ 表示选择(对应WHERE条件),  \\
- R 是关系名。)\\

\subsubsection{常见填空题考点:}
1. SQL的`SELECT`对应关系代数的 投影(π)。  

2. SQL的`WHERE`对应关系代数的 选择(σ)。

3. SQL的`JOIN`对应关系代数的 连接($\Join$)。  

4. SQL的`GROUP BY`对应关系代数的 分组(Γ)。

\subsection{关系范式}
\subsubsection{关系范式填空题模板}
\begin{enumerate}
  \item 关系模型应满足一定的规范要求,称为\underline{关系范式(或规范化)},其中最基本的三个范式是\underline{第一范式(1NF)}、\underline{第二范式(2NF)}、\underline{第三范式(3NF)}。
  
  \item 第一范式(1NF)要求关系模式中的每个属性的值域中\underline{不可再分},即每个属性都是\underline{原子值}。
  
  \item 第二范式(2NF)要求关系模式必须满足\underline{1NF},且非主属性完全函数依赖于\underline{候选键(或主键)}。
  
  \item 第三范式(3NF)要求关系模式必须满足\underline{2NF},且非主属性对候选键不存在\underline{传递函数依赖}。
  
  \item 巴斯-科德范式(BCNF)要求关系模式中的每一个函数依赖X→Y,决定因素X必须包含\underline{候选键(或超键)}。
  
  \item 对于关系模式R,若R∈3NF且消除了非平凡的多值依赖,则R属于\underline{第四范式(4NF)}。
  
  \item 第五范式(5NF)又称为\underline{投影-连接范式(Projection-Join Normal Form)},要求关系模式中不存在\underline{连接依赖}。
\end{enumerate}

\subsubsection{关键概念总结}
\begin{table}[htbp]
  \centering
  \caption{数据库范式核心条件}
  \label{tab:normal-forms}
  \begin{tabular}{lp{0.8\textwidth}}
  \toprule
  \textbf{范式} & \textbf{核心条件} \\
  \midrule
  1NF & 每个属性的值域不可再分(原子性)。 \\
  2NF & 满足1NF,且非主属性完全依赖于候选键。 \\
  3NF & 满足2NF,且非主属性不传递依赖于候选键。 \\
  BCNF & 满足3NF,且所有函数依赖的决定因素都是候选键。 \\
  4NF & 满足BCNF,且不存在非平凡的多值依赖。 \\
  5NF(PJNF/BC范式) & 满足4NF,且不存在非平凡的连接依赖。 \\
  \bottomrule
  \end{tabular}
  \end{table}
\subsubsection{常见考点}
\noindent 范式之间的关系:

每个更高范式都是前一个范式的超集(如:BCNF ⇒ 3NF ⇒ 2NF ⇒ 1NF)。

\noindent 典型反例:

违反2NF:如学生(学号,姓名,课程1,成绩1,课程2,成绩2),非主属性部分依赖主键。

违反3NF:如部门(部门号,部门名,经理,经理电话),经理电话传递依赖于部门号。

违反BCNF:如员工(项目,员工,技能),若项目→员工且员工→技能,则决定因素不包含候选键。

\noindent 规范化的作用:

消除数据冗余、插入异常、删除异常和更新异常。

\subsection{ 插入异常、修改异常(更新异常)和删除异常 }
\subsubsection{填空题模板}

\begin{enumerate}
    \item 插入异常是指\underline{\hspace{5cm}},通常由于关系模式中存在\underline{\hspace{2cm}}或\underline{\hspace{2cm}}导致。
    \begin{quote}
        \textcolor{blue}{答案:无法插入某些有意义的数据行;部分函数依赖;传递函数依赖}
    \end{quote}

    \item 修改异常(更新异常)是指\underline{\hspace{5cm}},例如修改某个属性值时需要\underline{\hspace{4cm}},容易导致数据不一致。
    \begin{quote}
        \textcolor{blue}{答案:修改数据时需要修改多处冗余数据;修改多个元组}
    \end{quote}

    \item 删除异常是指\underline{\hspace{5cm}},例如删除某个元组时可能同时丢失\underline{\hspace{4cm}}。
    \begin{quote}
        \textcolor{blue}{答案:删除元组时导致有用信息丢失;其他相关数据}
    \end{quote}

    \item 关系模式未规范化时,容易出现\underline{\hspace{2cm}}、\underline{\hspace{2cm}}和\underline{\hspace{2cm}},这些是关系规范化要解决的核心问题。
    \begin{quote}
        \textcolor{blue}{答案:插入异常;修改(更新)异常;删除异常}
    \end{quote}

    \item 若关系模式中存在非主属性对候选键的传递依赖,则可能导致\underline{\hspace{2cm}}和\underline{\hspace{2cm}}。
    \begin{quote}
        \textcolor{blue}{答案:插入异常;删除异常}
    \end{quote}

    \item 规范化到\underline{\hspace{4cm}}可以消除非主属性对候选键的传递依赖,从而解决部分异常。
    \begin{quote}
        \textcolor{blue}{答案:第三范式(3NF)}
    \end{quote}

    \item 在关系模式中,如果存在多值依赖,则可能引发\underline{\hspace{3cm}},需通过规范化到\underline{\hspace{3cm}}解决。
    \begin{quote}
        \textcolor{blue}{答案:插入异常;第四范式(4NF)}
    \end{quote}
\end{enumerate}

\subsubsection{关键概念总结}
\begin{table}[htbp]
    \centering
    \caption{数据库规范化异常类型表}
    \begin{tabular}{p{3cm}p{6cm}p{6cm}}
        \toprule
        \textbf{异常类型} & \textbf{定义} & \textbf{示例} \\
        \midrule
        插入异常 & 无法插入某些有意义的数据行,因主键或外键约束导致 & 无法插入"课程"表中尚未存在的课程信息,因需要关联学生表的主键 \\
        修改异常 & 修改数据时需要修改多处冗余数据,导致不一致或操作繁琐 & 修改"部门名称"需遍历所有员工记录,容易遗漏 \\
        删除异常 & 删除元组时导致有用信息丢失(如删除学生记录时同时丢失其选课信息) & 删除一个学生记录时,该学生的选课记录也被删除,但选课信息可能需要保留 \\
        \bottomrule
    \end{tabular}
\end{table}

\subsubsection{常见考点解析}
\begin{itemize}
    \item \textbf{异常产生的原因}:
    \begin{itemize}
        \item 插入异常:非主属性对主键存在部分依赖或传递依赖
        \item 修改异常:冗余数据(如重复存储的部门名称)需要同时修改多处
        \item 删除异常:主键关联的数据被级联删除,导致非主属性信息丢失
    \end{itemize}

    \item \textbf{规范化的作用}:
    \begin{itemize}
        \item 1NF到3NF:解决插入、修改、删除异常(通过消除非主属性的传递依赖)
        \item BCNF:进一步消除决定因素非键的函数依赖
        \item 4NF/5NF:解决多值依赖和连接依赖导致的异常
    \end{itemize}

    \item \textbf{典型反例}:
    \begin{itemize}
        \item 未规范化表:学生(学号,姓名,课程1,成绩1,课程2,成绩2)
        \begin{itemize}
            \item 插入异常:若学生未选满2门课程,无法插入
            \item 删除异常:删除学生记录时,所有课程信息被删除
        \end{itemize}
        \item 规范化后:拆分为学生(学号,姓名)和选课(学号,课程,成绩),消除异常
    \end{itemize}
\end{itemize}
\subsection{专门的关系运算,选择、投影、连接  、(除) }
\subsubsection{填空题模板}

\begin{enumerate}[itemsep=1.2em]
    \item 选择运算(Selection)的符号是\underline{\hspace{2cm}},其作用是\underline{\hspace{4cm}}。
    \begin{quote}
        \textcolor{blue}{答案:$\sigma$(西格玛);从关系中筛选满足条件的元组(行)。}
    \end{quote}

    \item 投影运算(Projection)的符号是\underline{\hspace{2cm}},其作用是\underline{\hspace{4cm}}。
    \begin{quote}
        \textcolor{blue}{答案:$\pi$(派);从关系中选择特定的属性(列),并消除重复元组。}
    \end{quote}

    \item 连接运算(Join)分为\underline{\hspace{3cm}}和\underline{\hspace{3cm}},其中自然连接要求\underline{\hspace{4cm}}。
    \begin{quote}
        \textcolor{blue}{答案:$\theta$连接(如等值连接、不等连接);自然连接;连接条件为公共属性的等值,并且自动去除重复列。}
    \end{quote}

    \item $\theta$连接的符号是\underline{\hspace{2cm}},其一般形式为$R$\underline{\hspace{2cm}}$S$。
    \begin{quote}
        \textcolor{blue}{答案:$\infty$(无穷符号);$\sigma_{\text{<条件>}}(R\times S)$,但通常直接写为$R \Join_{\text{<条件>}} S$。}
    \end{quote}

    \item 除运算(Division)的符号是\underline{\hspace{2cm}},其作用是\underline{\hspace{4cm}}。
    \begin{quote}
        \textcolor{blue}{答案:$\div$;用于查询满足“对于某个属性集的所有值,都存在对应的元组”的条件。}
    \end{quote}

    \item 自然连接要求两个关系有\underline{\hspace{3cm}},而$\theta$连接可以基于\underline{\hspace{4cm}}进行连接。
    \begin{quote}
        \textcolor{blue}{答案:公共属性;任意属性间的条件(如等值、不等值)。}
    \end{quote}

    \item 投影运算的结果\underline{\hspace{1cm}}(是/否)自动去重,因为关系模型中元组是\underline{\hspace{2cm}}的。
    \begin{quote}
        \textcolor{blue}{答案:是;无序且唯一。}
    \end{quote}

    \item 关系$R \div S$的结果是\underline{\hspace{4cm}},即$R$中满足\underline{\hspace{5cm}}的元组。
    \begin{quote}
        \textcolor{blue}{答案:$R$中在除数$S$的属性集上的所有组合均在$R$中存在对应元组;对于$S$的所有元组,$R$中存在元组与之匹配。}
    \end{quote}
\end{enumerate}

\subsubsection{关键概念总结}
\begin{table}[htbp]
    \centering
    \caption{关系代数运算对照表}
    \begin{tabular}{p{2.5cm}p{1.5cm}p{5cm}p{4.5cm}}
        \toprule
        \textbf{运算} & \textbf{符号} & \textbf{作用} & \textbf{示例} \\
        \midrule
        选择 & $\sigma$ & 筛选满足条件的元组(行) & $\sigma_{\text{年龄}>20}(R)$:选择年龄大于20的元组 \\
        投影 & $\pi$ & 提取指定属性(列),并自动去重 & $\pi_{\text{学号,姓名}}(\text{学生})$:提取学号和姓名列 \\
        连接 & $\Join$ & 合并两个关系的元组,基于条件(如等值或不等值) & $\text{学生} \Join \text{选课}$(学号相等)合并学生和选课表 \\
        除法 & $\div$ & 查询满足“对于$S$的所有元组,$R$中存在匹配”的元组 & $R \div S$:如查询选修了所有课程的学生 \\
        \bottomrule
    \end{tabular}
\end{table}

\subsubsection{常见考点解析}
\begin{itemize}
    \item \textbf{运算符号与作用}:
    \begin{itemize}
        \item 选择($\sigma$):行级过滤(\texttt{WHERE}子句的数学基础)
        \item 投影($\pi$):列级筛选(类似SQL的\texttt{SELECT}列)
        \item 连接($\Join$):表合并操作(对应SQL的\texttt{JOIN})
        \item 除法($\div$):复杂查询(如“所有X都满足Y”)
    \end{itemize}

    \item \textbf{运算特点}:
    \begin{itemize}
        \item 投影自动去重:关系模型要求元组唯一
        \item 自然连接:自动去除重复列(如两个表的公共属性只保留一列)
        \item 除法条件:$R \div S$的结果是$R$中在$S$的属性集上的投影
    \end{itemize}

    \item \textbf{典型反例}:
    \begin{itemize}
        \item 除法应用示例:
        \begin{itemize}
            \item $R$(学号,课程)和$S$(课程)
            \item $R \div S$的结果是“选修了$S$中所有课程的学生学号”
            \item 若$S=\{\text{C1},\text{C2}\}$,则结果需同时包含C1和C2的学号
        \end{itemize}
    \end{itemize}
\end{itemize}
\section{简答题 30points}
\subsection{数据挖掘步骤}
数据挖掘的典型步骤包括:

业务理解(明确目标);

数据理解(探索数据);

数据准备(清洗、转换数据);

建立模型(选择算法建模);

模型评估(验证性能);

部署应用(实际落地)。
\subsection{文件系统阶段的缺陷}
文件系统阶段的主要缺陷包括:

数据冗余与不一致性(更新困难);

数据共享性差(文件格式不统一);

数据独立性低(程序与数据强耦合);

安全性不足(权限控制粗放);

缺乏完整性控制;

并发控制问题;

查询效率低下;

缺乏统一管理。
\subsection{视图与基本表的联系与区别}
\begin{enumerate}
    \item \textbf{联系}:
    \begin{enumerate}
        \item 视图基于基本表定义,依赖其数据;
        \item 修改基本表会影响视图结果,视图更新(若允许)会反向修改基本表。
    \end{enumerate}
    
    \item \textbf{区别}:
    \begin{enumerate}
        \item 存储方式:
            \begin{itemize}
                \item 基本表:存储真实数据;
                \item 视图:仅存储查询定义。
            \end{itemize}
        \item 修改权限:
            \begin{itemize}
                \item 基本表:支持直接增删改查;
                \item 视图:默认仅允许查询,更新受限制。
            \end{itemize}
        \item 安全控制:
            \begin{itemize}
                \item 视图可隐藏敏感字段,实现数据访问权限控制。
            \end{itemize}
        \item 独立性:
            \begin{itemize}
                \item 视图隔离底层表结构变化,提升应用逻辑独立性。
            \end{itemize}
    \end{enumerate}
\end{enumerate}
\subsection{系统故障的恢复策略}
系统故障的恢复策略包括:

日志记录:记录事务操作,支持回滚和重做;

检查点机制:定期将内存数据写入磁盘,减少恢复范围;

UNDO(回滚) :撤销未提交事务的修改;

REDO(重做) :重做已提交但未持久化的事务;

预写日志(WAL) :确保数据与日志的顺序一致性。
\subsection{调度封锁}
\subsection{数据库管理系统存取数据的流程}
数据库管理系统存取数据的流程包括:

\begin{enumerate}
    \item 用户请求与解析(语法、语义检查);
    \item 查询优化(选择访问路径);
    \item 执行引擎(执行计划);
    \item 事务管理(并发控制、日志记录);
    \item 存储引擎(访问物理数据);
    \item 缓冲区与持久化(脏页刷新、检查点);
    \item 结果返回。
\end{enumerate}
\subsection{事务隔离级别的区别}
\subsubsection{核心区别总结}

\begin{table}[h!]
    \centering
    \caption{隔离级别之间的核心区别}
    \begin{tabular}{@{}>{\centering\arraybackslash}m{4cm}%
                        >{\centering\arraybackslash}m{1.5cm}%
                        >{\centering\arraybackslash}m{2.5cm}%
                        >{\centering\arraybackslash}m{2cm}%
                        >{\centering\arraybackslash}m{2.5cm}%
                        >{\centering\arraybackslash}m{3.5cm}@{}}
        \toprule
        \textbf{隔离级别} & \textbf{脏读} & \textbf{不可重复读} & \textbf{幻读} & \textbf{并发性能} & \textbf{典型实现} \\
        \midrule
        读未提交           & \checkmark   & \checkmark        & \checkmark   & 高              & 无锁机制              \\
        读已提交           & $\times$        & \checkmark        & \checkmark   & 较高            & 锁或MVCC(部分场景)  \\
        可重复读           & $\times$        & $\times$            & \checkmark   & 中等            & MVCC(如MySQL)      \\
        串行化             & $\times$        & $\times$            & $\times$        & 低              & 排他锁              \\
        \bottomrule
    \end{tabular}
    
\end{table}

\section*{关键问题的定义}

\begin{enumerate}
    \item \textbf{脏读(Dirty Read)}:读取到其他事务未提交的数据,若该事务回滚,则数据无效。
    \item \textbf{不可重复读(Non-Repeatable Read)}:同一事务内多次读取同一数据,结果不同(因其他事务已提交更新)。
    \item \textbf{幻读(Phantom Read)}:同一事务内多次查询同一条件的数据,结果集行数或内容变化(因其他事务插入了符合条件的新数据)。
\end{enumerate}
\subsection{事务的AICD}
\subsubsection{ACID特性总结}

\begin{table}[h!]
    \centering
    \caption{ACID特性总结}
    \begin{tabular}{@{}>{\centering\arraybackslash}m{3cm}%
                        >{\arraybackslash}m{6cm}%
                        >{\arraybackslash}m{6cm}@{}}
        \toprule
        \textbf{特性}   & \textbf{核心作用}         & \textbf{典型场景}                \\
        \midrule
        \textbf{原子性} & 避免“半完成”操作         & 银行转账、订单创建               \\
        \textbf{一致性} & 保证数据合法性和业务规则 & 数据库约束(如主键、外键)       \\
        \textbf{隔离性} & 避免并发问题(如脏读、幻读) & 多用户同时操作同一数据           \\
        \textbf{持久性} & 数据提交后永不丢失       & 系统崩溃后恢复数据               \\
        \bottomrule
    \end{tabular}
    
\end{table}

\subsubsection{关键问题}

\begin{enumerate}
    \item \textbf{为什么需要ACID?}
    \begin{itemize}
        \item 保证事务的可靠性,确保数据在并发、故障等复杂环境下仍能正确处理。
    \end{itemize}
    \item \textbf{ACID如何实现?}
    \begin{itemize}
        \item 通过日志(保证原子性和持久性)、锁/MVCC(保证隔离性)、约束(保证一致性)等机制。
    \end{itemize}
    \item \textbf{ACID与BASE(NoSQL的软状态模型)的区别?}
    \begin{itemize}
        \item ACID强调强一致性,BASE强调最终一致性(如高并发场景下牺牲强一致性换取可用性)。
    \end{itemize}
\end{enumerate}

\subsubsection{备考建议}
\begin{itemize}
    \item \textbf{记忆技巧}:用“\textbf{原子、一致、隔离、持久}”四字口诀记忆ACID。
    \item \textbf{结合实例}:用银行转账、订单系统等经典案例理解每个特性的作用。
    \item \textbf{关联隔离级别}:隔离性(Isolation)与事务的四个隔离级别(如Read Committed)直接相关,需结合理解。
\end{itemize}
\subsection{完整性约束}
\subsubsection{约束类型与其核心作用}

\begin{table}[h!]
    \centering
    \caption{数据库约束类型总结}
    \begin{tabular}{@{}>{\centering\arraybackslash}m{4cm}%
                        >{\arraybackslash}m{5cm}%
                        >{\arraybackslash}m{6cm}@{}}
        \toprule
        \textbf{约束类型}     & \textbf{核心作用}         & \textbf{典型场景}                \\
        \midrule
        \textbf{实体完整性}   & 确保主键唯一且非空       & 学生表的学号、订单表的订单号     \\
        \textbf{域完整性}     & 限制字段的取值范围和类型 & 年龄限制、成绩范围               \\
        \textbf{参照完整性}   & 保证外键引用有效         & 订单与客户表关联、部门与员工关联 \\
        \textbf{用户定义完整性} & 实现业务规则(如折扣率限制) & 价格区间、性别选项               \\
        \textbf{唯一性约束}   & 确保字段值唯一(允许空值) & 邮箱、身份证号                   \\
        \textbf{默认值约束}   & 自动填充字段的默认值     & 创建时间、状态标志               \\
        \bottomrule
    \end{tabular}
    
\end{table}

\subsubsection{备考建议}

\begin{enumerate}
    \item \textbf{记忆关键点}:用“\textbf{实体、域、参照、用户定义}”四类约束框架记忆。
    \item \textbf{结合SQL语法}:掌握 \texttt{PRIMARY KEY}、\texttt{FOREIGN KEY}、\texttt{CHECK}、\texttt{UNIQUE} 等约束的SQL实现。
    \item \textbf{案例分析}:通过实际表设计(如学生选课系统)理解约束的应用场景。
    \item \textbf{常见考点}:
    \begin{itemize}
        \item 不同约束的区别(如主键与唯一性约束)。
        \item 违反约束的后果及处理方式。
        \item 约束对数据一致性的保障作用。
    \end{itemize}
\end{enumerate}
\subsection{数据库的优化}
\subsubsection{优化原则}

\begin{table}[h!]
    \centering
    \caption{数据库优化原则}
    \begin{tabular}{@{}>{\centering\arraybackslash}m{7cm}%
                        >{\arraybackslash}m{8cm}@{}}
        \toprule
        \textbf{优化原则} & \textbf{具体说明} \\
        \midrule
        \textbf{先分析后优化} & 
        通过慢查询日志、性能监控工具(如`SHOW PROCESSLIST`、`pt-query-digest`)定位瓶颈。\\
        \textbf{避免过度优化} &
        优先解决核心问题,避免为小问题引入复杂方案。\\
        \textbf{平衡读写} &
        根据业务场景权衡一致性与性能(如OLTP vs OLAP)。\\
        \bottomrule
    \end{tabular}
    
\end{table}

\subsubsection{简答题答题要点}

\begin{enumerate}
    \item \textbf{核心方法}:
    \begin{itemize}
        \item 索引优化
        \item 查询优化
        \item 存储优化
        \item 并发控制
    \end{itemize}
    \item \textbf{关键工具}:
    \begin{itemize}
        \item `EXPLAIN`
        \item 慢查询日志
        \item 执行计划分析
    \end{itemize}
    \item \textbf{典型场景}:
    \begin{itemize}
        \item 分页优化
        \item 减少全表扫描
        \item 合理分区
        \item 避免死锁
    \end{itemize}
\end{enumerate}
\subsection{关系模型和网状模型的数据结构}

\subsubsection{核心区别对比}

\begin{table}[h!]
    \centering
    \caption{关系模型与网状模型的核心区别}
    \begin{tabular}{@{}>{\centering\arraybackslash}m{4cm}%
                        >{\arraybackslash}m{6cm}%
                        >{\arraybackslash}m{6cm}@{}}
        \toprule
        \textbf{特征}            & \textbf{关系模型}                          & \textbf{网状模型}                          \\
        \midrule
        \textbf{数据结构}        & 二维表格(关系)                            & 有向图(记录+指针)                        \\
        \textbf{关联方式}        & 主键/外键(显式声明)                       & 指针(隐式地址引用)                       \\
        \textbf{查询语言}        & SQL(声明式)                               & 网状语言(过程式)                         \\
        \textbf{标准化}          & 高(支持范式)                              & 低(依赖指针,冗余可能高)                 \\
        \textbf{灵活性}          & 查询灵活(通过SQL)                         & 查询复杂(需指针导航)                     \\
        \textbf{维护成本}        & 低(结构化约束)                            & 高(依赖指针路径)                         \\
        \textbf{适用场景}        & 结构化、事务性数据                          & 复杂多对多关系、资源受限环境              \\
        \bottomrule
    \end{tabular}
    
\end{table}

\subsubsection{关键问题总结}

\begin{enumerate}
    \item \textbf{关系模型的优势}:
    \begin{itemize}
        \item 简单直观,支持标准SQL,易于维护。
        \item 符合ACID,适合高并发事务场景。
    \end{itemize}
    \item \textbf{网状模型的局限性}:
    \begin{itemize}
        \item 结构复杂,查询需依赖指针路径。
        \item 扩展性差,修改结构需调整大量指针。
    \end{itemize}
    \item \textbf{历史地位}:
    \begin{itemize}
        \item 网状模型是关系模型的前身,已被关系模型取代(如IBM的IMS系统逐渐过渡到DB2)。
    \end{itemize}
\end{enumerate}

\subsubsection{考研备考建议}

\begin{enumerate}
    \item \textbf{重点记忆}:
    \begin{itemize}
        \item 二维表 vs 网状图
        \item 主键/外键 vs 指针
        \item SQL vs 过程式查询
    \end{itemize}
    \item \textbf{对比分析}:
    \begin{itemize}
        \item 通过表格对比结构、关联方式、查询语言、适用场景等。
    \end{itemize}
    \item \textbf{案例理解}:
    \begin{itemize}
        \item 关系模型:学生-课程表通过学号关联。
        \item 网状模型:部门记录通过指针直接指向多个员工记录。
    \end{itemize}
\end{enumerate}
\section{大题 50points}
\subsection{SQL语句 30 points/8}
考点包括:创建视图、修改表结构、插入数据、删除数据、group by语句
\subsection{E-R图设计 12 points}
\subsection{函数依赖-范式 8 points}
\end{document}