\section{数据库恢复技术}
\subsection{事务的概念}
\begin{enumerate}
    \item 数据库操作序列:定义事务为一组相关数据库操作的集合,作为一个整体执行。
    \item 恢复的基本单位和并发控制的基本单位:确保在故障恢复时保持数据一致性和完整性。
\end{enumerate}

\subsection{事务的SQL语句}
% 更新:扩展SQL语句,加入提交和保存点等相关命令
\begin{itemize}
    \item 提交:COMMIT
    \item 回滚:ROLLBACK
    \item 保存点:SAVEPOINT
\end{itemize}

\subsection{事务的四个特性}
\begin{enumerate}
    \item 原子性
    \item 一致性
    \item 隔离性
    \item 持续性
\end{enumerate}

\subsection{DBS的故障种类}
\begin{enumerate}
    \item 事务内部的故障
    \item 系统故障(软故障,如软件错误)
    \item 介质故障(硬故障,如硬盘损坏)
    \item 计算机病毒及其他安全问题
\end{enumerate}

\subsection{数据库恢复技术}
\begin{enumerate}
    \item 数据转储:采用全量备份与增量备份策略,便于恢复最新数据。
    \item 日志记录:登记日志文件,既可按记录为单位,也可按数据块为单位,辅助精确恢复。
\end{enumerate}

