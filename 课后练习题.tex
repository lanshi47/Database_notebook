% !TEX program = xelatex
\documentclass[a4paper,12pt,UTF8]{ctexart}
\usepackage{geometry} % 页面设置
\usepackage{xcolor} % Color support for listings
\usepackage{graphicx}
\usepackage{amssymb} % For join symbol
\usepackage{amsmath}
\usepackage{listings}
% 设置代码块样式
\lstset{
    basicstyle=\ttfamily,
    columns=fullflexible,
    frame=single,
    breaklines=true,
    postbreak=\mbox{\textcolor{red}{$\hookrightarrow$}\space},
    language=SQL,
    keywordstyle=\color{blue},
    commentstyle=\color{gray},    
    rulecolor=\color{black!30},%边框颜色
    stringstyle=\color{red},
    escapeinside={\%*}{*)},
    showstringspaces=false,
    captionpos=b % 设置标题位置, b表示在底部
}
\geometry{left=2.5cm,right=2.5cm,top=2.5cm,bottom=2.5cm} % 页边距

% 封面设置
\title{SQL数据库系统课后练习}
\author{lanshi}
\date{\today}

\begin{document}

% 制作封面页
\begin{titlepage}
    \centering
    \vspace*{\fill}
    {\LARGE\bfseries 数据库系统原理课程\par}
    \vspace{2cm}
    {\Huge\bfseries SQL课后练习题集\par}
    \vspace{2cm}
    {\Large 烂石\par}
    \vspace{1cm}
    {\large \today \par}
    \vspace{4cm}
    \includegraphics[width=0.5\textwidth]{image/logo.jpg}
    \vspace*{\fill}
    \thispagestyle{empty} % 封面不显示页码
    \newpage
\end{titlepage}

% 正文内容
\section{ER图相关习题}
\section{关系代数运算相关习题}
例题:现有关系S(S\#,SNAME,AGE,SEX)),C(C\#,CNAME,TEACHER)和SC(S\#,C\#,GRADE),试用表达式表示以下查询语句:
第一个问题:查询至少选修"程军"老师所授全部课程的学生姓名(SNAME);
解析:有三个部分,要查询"程军"老师的全部课程;要查询学生的选课记录,包括学号和课程号;要查询学生的姓名.
至少表示要查询选修了全部课程的学生,即选修了"程军"老师的全部课程的学生.
所以,首先要找到"程军"老师的全部课程,然后找到选修了这些课程的学生,最后找到这些学生的姓名.
这个查询可以分为三个部分:
1.找到"程军"老师的全部课程:\\
\begin{equation}
    \pi_{(C\#(\sigma_(TEACHER='\text{程军}')(C)))}
\end{equation}
2.学生选课记录:\\
\begin{equation}
    \pi_{S\# C\#(SC)}
\end{equation}
3.筛选学生:
\begin{equation}
    {\pi_{S\#,C\#(SC)}}\div{\pi_{(C\#(\sigma_(TEACHER='\text{程军}')(C)}}
\end{equation}
综合以上三个部分,可以得到整个查询的表达式:\\
\begin{equation}
    \pi_{\text{SNAME}} \Big( S \Join \big( \pi_{\text{S\#, C\#}}(SC) \div \pi_{\text{C\#}}( \sigma_{\text{TEACHER='程军'}}(C) ) \big) \Big)
\end{equation}
 
\section{SQL语句}
\textbf{题1:}\\
1. 设学生课程数据库中有三个关系:

   学生关系 S (S\#, SNAME, AGE, SEX)
    学习关系 SC (S\#, C\#, GRADE)
   课程关系 C (C\#, CNAME)

   其中 S\#, C\#, SNAME, AGE, SEX, GRADE, CNAME 分别表示学号、课程号、姓名、年龄、性别、成绩和课程名。

用 SQL 语句表达以下操作:

1. 检索选修课程名称为 "MATHS" 的学生学号与姓名。
答:\begin{lstlisting}
    SELECT DISTINCT S.S#, S.SNAME
    FROM S
    JOIN SC ON S.S# = SC.S#
    JOIN C ON SC.C# = C.C#
    WHERE C.CNAME = 'MATHS';
\end{lstlisting}
2. 检索至少学习了课程号为 "C1" 和 "C2" 的学生的学号。
\begin{lstlisting}
    SELECT S# FROM SC
    WHERE C# IN("C1","C2")
    GROUP BY S#
    HAVING COUNT (DISTINCT C#)=2
\end{lstlisting}
3. 检索年龄在 18 到 20 之间(含 18 和 20)的女性学生的学号、姓名和年龄。
\begin{lstlisting}
    SELECT S#, SNAME, AGE
    FROM S
    WHERE AGE BETWEEN 18 AND 20
      AND SEX = '女'; 
\end{lstlisting}
4. 检索平均成绩达到 80 的学生学号和平均成绩。
\begin{lstlisting}
    SELECT S#,AVG(GRADE) AS AVG_GRADE
    FROM SC
    GROUP BY S#
    HAVING AVG(GRADE)>=80;
\end{lstlisting}
5. 检索选修了全部课程的学生姓名。
\begin{lstlisting}
    SELECT S.SNAME
    FROM S 
    WHERE NOT EXISTS(
        SELECT C.C#
        FROM C 
        WHERE S# NOT IN(
            SELECT *
            FROM SC
            WHERE S.S#=SC.S#
            AND C.C#=SC.C#
        )
    )
\end{lstlisting}
6. 检索选修了三个课程以上的学生的学号。
\begin{lstlisting}
    SELECT S#
    FROM SC
    GROUP BY S#
    HAVING COUNT(DISTINCT C#) > 3 
\end{lstlisting}
\textbf{题2:学生-课程数据库中包括三个表:}\\
\begin{itemize}
    \item 学生表:\textbf{Student} (\texttt{Sno}, \texttt{Sname}, \texttt{Sex}, \texttt{Sage}, \texttt{Sdept})
    \item 课程表:\textbf{Course} (\texttt{Cno}, \texttt{Cname}, \texttt{Ccredit})
    \item 学生选课表:\textbf{SC} (\texttt{Sno}, \texttt{Cno}, \texttt{Grade})
\end{itemize}

其中 \texttt{Sno}、\texttt{Sname}、\texttt{Sex}、\texttt{Sage}、\texttt{Sdept}、\texttt{Cno}、\texttt{Cname}、\texttt{Ccredit}、\texttt{Grade} 分别表示学号、姓名、性别、年龄、所在系名、课程号、课程名、学分和成绩。

\subsection*{试用 SQL 语言完成下列操作:}

\begin{enumerate}
    \item 查询选修课程包括 “1042” 号学生所学的课程的学生学号。\\
    \begin{lstlisting}
        SELECT DISTINCT Sno
        FROM SC AS X
        WHERE NOT EXISTS (
            SELECT Cno
            FROM SC
            WHERE Sno = '1042'    -- 获取1042学生的所有课程
            AND Cno NOT IN (       -- 检查是否存在1042选修的课程未被当前学生选修
                SELECT Cno
                FROM SC AS Y
                WHERE Y.Sno = X.Sno
            )
        );
    \end{lstlisting}
    \item 创建一个计算系学生信息视图 \textbf{CS\_VIEW},包括 \texttt{Sno} 学号、\texttt{Sname} 姓名、\texttt{Sex} 性别。
    \begin{lstlisting}
        CREATE VIEW CS_VIEW AS
        SELECT Sno,Sname,Sex
        FROM Student
        WHERE Sdept='计算系'
    \end{lstlisting}
    \item 通过上面第 2 题创建的视图修改数据,把王平的名字改为王慧平。
    \begin{lstlisting}
        UPDATE CS_VIEW
        SET Sname='王慧平'
        WHERE Sname='王平'
    \end{lstlisting}
    \item 创建一选修数据库课程信息的视图,视图名称为 \textbf{datascore\_view},包含学号、姓名、成绩。
    \begin{lstlisting}
        CREATE VIEW datascore_view AS
        SELECT Student.Sno, Sname, Grade
        FROM Student
        JOIN SC ON Student.Sno = SC.Sno
        JOIN Course ON SC.Cno = Course.Cno
        WHERE Course.Cname = '数据库';
    \end{lstlisting}
\end{enumerate}
\end{document}