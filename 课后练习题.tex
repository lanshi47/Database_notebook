% !TEX program = xelatex
\documentclass[12pt,a4paper]{article}
\usepackage[UTF8]{ctex} % 中文支持
\usepackage{geometry} % 页面设置
\usepackage{graphicx}
\usepackage{amssymb} % For join symbol
\usepackage{amsmath}
\geometry{left=2.5cm,right=2.5cm,top=2.5cm,bottom=2.5cm} % 页边距

% 封面设置
\title{SQL数据库系统课后练习}
\author{lanshi}
\date{\today}

\begin{document}

% 制作封面页
\begin{titlepage}
    \centering
    \vspace*{\fill}
    {\LARGE\bfseries 数据库系统原理课程\par}
    \vspace{2cm}
    {\Huge\bfseries SQL课后练习题集\par}
    \vspace{2cm}
    {\Large 烂石\par}
    \vspace{1cm}
    {\large \today \par}
    \vspace{4cm}
    \includegraphics[width=0.5\textwidth]{image/logo.jpg}
    \vspace*{\fill}
    \thispagestyle{empty} % 封面不显示页码
    \newpage
\end{titlepage}

% 正文内容
\section{ER图相关习题}
\section{关系代数运算相关习题}
例题:现有关系S(S\#,SNAME,AGE,SEX)),C(C\#,CNAME,TEACHER)和SC(S\#,C\#,GRADE),试用表达式表示以下查询语句:
第一个问题:查询至少选修"程军"老师所授全部课程的学生姓名(SNAME);
解析:有三个部分,要查询"程军"老师的全部课程;要查询学生的选课记录,包括学号和课程号;要查询学生的姓名.
至少表示要查询选修了全部课程的学生,即选修了"程军"老师的全部课程的学生.
所以,首先要找到"程军"老师的全部课程,然后找到选修了这些课程的学生,最后找到这些学生的姓名.
这个查询可以分为三个部分:
1.找到"程军"老师的全部课程:\\
\begin{equation}
    \pi_{(C\#(\sigma_(TEACHER='\text{程军}')(C)))}
\end{equation}
2.学生选课记录:\\
\begin{equation}
    \pi_{S\# C\#(SC)}
\end{equation}
3.筛选学生:
\begin{equation}
    {\pi_{S\#,C\#(SC)}}\div{\pi_{(C\#(\sigma_(TEACHER='\text{程军}')(C)}}
\end{equation}
综合以上三个部分,可以得到整个查询的表达式:\\
\begin{equation}
    \pi_{\text{SNAME}} \Big( S \Join \big( \pi_{\text{S\#, C\#}}(SC) \div \pi_{\text{C\#}}( \sigma_{\text{TEACHER='程军'}}(C) ) \big) \Big)
\end{equation}


\end{document}