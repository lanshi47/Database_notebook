% !TEX program = xelatex
\documentclass[12pt,a4paper]{article}
\usepackage[UTF8]{ctex} % 中文支持
\usepackage{geometry} % 页面设置
\usepackage{graphicx}
\geometry{left=2.5cm,right=2.5cm,top=2.5cm,bottom=2.5cm} % 页边距

% 封面设置
\title{SQL数据库系统课后练习}
\author{lanshi}
\date{\today}

\begin{document}

% 制作封面页
\begin{titlepage}
    \centering
    \vspace*{\fill}
    {\LARGE\bfseries 数据库系统原理课程\par}
    \vspace{2cm}
    {\Huge\bfseries SQL课后练习题集\par}
    \vspace{2cm}
    {\Large 烂石\par}
    \vspace{1cm}
    {\large \today \par}
    \vspace{4cm}
    \includegraphics[width=0.5\textwidth]{image/logo.jpg}
    \vspace*{\fill}
    \thispagestyle{empty} % 封面不显示页码
    \newpage
\end{titlepage}

% 正文内容
\section{基础练习题}
\begin{enumerate}
    \item 请写出创建学生表的SQL语句,要求包含以下字段:
    \begin{itemize}
        \item 学号(主键,字符型,长度10)
        \item 姓名(非空,字符型,长度20)
        \item 性别(枚举类型,只能输入'男'或'女')
        \item 出生日期(日期类型)
        \item 所在院系(字符型,长度30)
    \end{itemize}
    
    \vspace{3cm} % 预留答题空间
    
    \item 编写查询语句:显示所有计算机学院学生的姓名和出生日期,按出生日期从早到晚排序。
    
    \vspace{3cm}
    
    \item 解释以下SQL关键词的区别:
    \begin{itemize}
        \item WHERE vs HAVING
        \item INNER JOIN vs LEFT JOIN
        \item DELETE vs TRUNCATE
    \end{itemize}
    
    \vspace{3cm}
\end{enumerate}

\section{进阶练习题}
\begin{enumerate}
    \item 设计一个包含课程表和选课表的关系模式,并建立适当的约束关系
    
    \vspace{5cm}
    
    \item 编写存储过程:统计指定院系的学生人数,并返回统计结果
    
    \vspace{5cm}
\end{enumerate}

\end{document}