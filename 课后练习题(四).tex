\documentclass[a4paper,12pt,UTF8,fontset=none]{ctexart}
\usepackage{geometry} % 页面设置
\usepackage{xeCJK}
\usepackage{enumitem}
\usepackage{xcolor} % Color support for listings
\usepackage{graphicx}
\usepackage{float}
\usepackage{pdfpages}
\usepackage{amssymb} % For join symbol
\usepackage{amsmath}
\usepackage{listings}
\usepackage{longtable}
\usepackage{booktabs} % 添加booktabs宏包以支持三线表
\usepackage{tabularx} % 新增tabularx宏包用于自动调整列宽
\usepackage{array}      % 列格式控制
\usepackage{caption}    % 表格标题格式
\usepackage{placeins} 
\usepackage{newtxtext,newtxmath}
\usepackage{fontspec}
\usepackage{titlesec}
\usepackage{cleveref}
\usepackage{background} % 添加水印宏包
\backgroundsetup{ % 水印参数设置
    scale=3,          % 水印文字大小(可调)
    angle=0,         % 水印旋转角度
    opacity=0.25,     % 透明度(0-1)
    contents={\includegraphics[width=0.5\textwidth]{static/pdfs/水印.pdf}} % 水印文件
}
% 设置英文主字体为 Times New Roman
\setmainfont{Times New Roman}[Path=D:/Program Files/MiKTeX/fonts/custom/, Extension=.otf]

% 设置英文粗体字体为 Times New Roman Bold
\newfontfamily\enboldfont{Times New Roman Bold}[Path=D:/Program Files/MiKTeX/fonts/custom/, Extension=.otf]

% 设置中文正文字体为宋体
\setCJKmainfont{SimSun}[Path=D:/Program Files/MiKTeX/fonts/custom/, Extension=.ttf]

% 设置中文黑体字体
\setCJKfamilyfont{zhhei}{SimHei}[Path=D:/Program Files/MiKTeX/fonts/custom/, Extension=.ttf]
\newcommand{\heiti}{\CJKfamily{zhhei}}

% 自定义编号格式
\renewcommand{\thesection}{\arabic{section}}
\renewcommand{\thesubsection}{\thesection.\arabic{subsection}}
\renewcommand{\thesubsubsection}{\thesubsection.\arabic{subsubsection}}
% 设置标题格式
\titleformat{\section}
  {\centering\heiti\zihao{-3}\enboldfont}{\thesection}{1em}{}
\titleformat{\subsection}
  {\heiti\zihao{4}\enboldfont}{\thesubsection}{1em}{}
\titleformat{\subsubsection}
  {\heiti\zihao{-4}\enboldfont}{\thesubsubsection}{1em}{}


% 定义计数器
\newcounter{mycounter}

% 定义命令用于显示编号
\newcommand{\myitem}{%
  \par \stepcounter{mycounter}%
  \textbf{\themycounter.}~%
}
% 在每个 section 开始时重置计数器
\let\oldsection\section
\renewcommand{\section}{%
  \setcounter{mycounter}{0}%
  \oldsection%
}
% 设置代码块样式
\lstset{
    basicstyle=\ttfamily,
    columns=fullflexible,
    frame=single,
    breaklines=true,
    postbreak=\mbox{\textcolor{red}{$\hookrightarrow$}\space},
    language=SQL,
    keywordstyle=\color{blue},
    commentstyle=\color{gray},    
    rulecolor=\color{black!30},%边框颜色
    stringstyle=\color{red},
    escapeinside={\%*}{*)},
    showstringspaces=false,
    captionpos=b % 设置标题位置, b表示在底部
}
\geometry{left=2.5cm,right=2.5cm,top=2.5cm,bottom=2.5cm} % 页边距


\begin{document}

% 制作封面页
\begin{titlepage}
    \centering
    \vspace*{\fill}
    {\Large\bfseries 数据库系统原理课程\par}
    \vspace{2cm}
    {\LARGE\bfseries 数据库系统考试\par}
    \vspace{2cm}
    {\Large\bfseries 查漏补缺-综合\par}
    \vspace{2cm}
    {\Large 烂石\par}
    \vspace{1cm}
    {\large \today \par}
    \vspace{4cm}
    \includegraphics[width=0.5\textwidth]{static/images/logo.jpg}
    \vspace*{\fill}
    \thispagestyle{empty} % 封面不显示页码
    \newpage
\end{titlepage}
\section{绪论}
\myitem 数据库的核心是\underbar{\color{red}DBMS}
\myitem 数据库系统最大的特点是\underline{\color{red}数据的三级抽象和二级独立性}
\myitem DBMS是位于\underline{\color{red}用户和操作系统}之间的一层管理软件
\myitem 数据模型是由\underbar{\color{red}数据结构,数据操作和完整性约束}三部分组成的
\myitem \underline{\color{red}数据结构}是对数据系统的静态特性描述,\underline{\color{red}数据操作}是动态特性描述
\myitem 什么是数据库?
\par 答:数据库是长期存储在计算机内、有组织的、可共享的数据集合。数据库是按某种数据模型进行组织的、存放在外存储器上,且可被多个用户同时使用。因此,数据库具有较小的冗余度,较高的数据独立性和易扩展性。
\myitem 什么是数据库的数据独立性?

    答:数据独立性表示应用程序与数据库中存储的数据不存在依赖关系,包括逻辑数据独立性和物理数据独立.
    逻辑数据独立性是指局部逻辑数据结构(外视图即用户的逻辑文件)与全局逻辑数据结构(概念视图)之间的独立性。当数据库的全局逻辑数据结构(概念视图)发生变化(数据定义的修改、数据之间联系的变更或增加新的数据类型等)时,它不影响某些局部的逻辑结构的性质,应用程序不必修改。
    物理数据独立性是指数据的存储结构与存取方法(内视图)改变时,对数据库的全局逻辑结构(概念视图)和应用程序不必作修改的一种特性,也就是说,数据库数据的存储结构与存取方法独立。
\myitem 什么是数据库管理系统?

    答:数据库管理系统(DBMS)是操纵和管理数据库的一组软件,它是数据库系统(DBS)的重要组成部分。不同的数据库系统都配有各自的DBMS,而不同的DBMS各支持一种数据库模型,虽然它们的功能强弱不同,但大多数DBMS的构成相同,功能相似。一般说来,DBMS具有定义、建立、维护和使用数据库的功能,它通常由三部分构成:数据描述语言及其翻译程序、数据操纵语言及其处理程序和数据库管理的例行程序。
\myitem 数据库设计一般分为哪几个阶段,简述每个阶段的主要任务是什么?
    
解:(1)数据库设计分为6个阶段:需求分析、概念结构设计、逻辑结构设计、物理结构设计、数据库实施、数据库运行和维护。
(2)各阶段任务如下:
①需求分析:准确了解与分析用户需求(包括数据与处理)。
②概念结构设计:通过对用户需求进行综合、归纳与抽象,形成一个独立于具体 DBMS 的概念模型。
③逻辑结构设计:将概念结构转换为某个 DBMS 所支持的数据模型,并对其进行优化。
④数据库物理设计:为逻辑数据模型选取一个最适合应用环境的物理结构(包括存储结构和存取方法)。
⑤数据库实施:设计人员运用 DBMS 提供的数据语言、工具及宿主语言,根据逻辑设计和物理设计的结果建立数据库,编制与调试应用程序,组织数据入库,并进行试运行。 
⑥数据库运行和维护:在数据库系统运行过程中对其进行评价、调整与修改。
\section{关系数据库理论}
\myitem 关系模式定义格式为\underline{关系名(属性名1,,,,属性名n)}
\myitem 关系模式的定义包括\underline{关系名,属性名,属性类型,属性长度,关键字}
\myitem 关系数据库中的基于数学上两类运算是\underline{关系代数}和\underline{关系演算}
\myitem 传统集合运算有\underline{差,交,并,笛卡尔积}
\myitem 专门的关系运算有\underline{连接,投影,选择}
\section{其他}
\myitem SQL变量声明与赋值的语法

变量声明:在SQL Server中,变量必须以 @ 开头,使用 DECLARE 关键字声明,并必须指定数据类型。
语法:
\begin{lstlisting}
  DECLARE @变量名 数据类型;
\end{lstlisting}
变量赋值:可以通过 SELECT 或 SET 语句赋值。
\begin{lstlisting}
  DECLARE @count INT;
SELECT @count = 1;  -- 或使用 SET @count = 1;
\end{lstlisting}
\begin{itemize}
  \item 临时存储中间结果,减少重复查询。
  \item 控制流程逻辑(如循环、条件判断)。
  \item  动态构建SQL语句,增强灵活性。
  \item  存储过程参数化,提升代码复用性。
\end{itemize}
\myitem 删除存储过程用\underbar{drop proc}
\myitem 多个语句捆绑用\underline{BEGIN-END}
\myitem REVOKE语法--列级权限收回
\begin{lstlisting}
  REVOKE UPDATE (列名) ON 表名 FROM 用户名;
\end{lstlisting}
\par  收回权限--特定列的修改权
\begin{lstlisting}
  GRANT UPDATE (列名) ON 表名 TO 用户名;
\end{lstlisting}
\myitem  释放游标资源用\underbar{DELALLOCATE}
\myitem 保护数据安全性的一般方法是\underline{设置用户标识}和\underline{存取控制权限}
\myitem 安全性控制的一般方法有\underline{\color{red}用户标识鉴定,存取控制,审计,视图,数据加密}的保护五级安全措施
\myitem 存取权限包括两方面的内容,一个是\underline{要存取的数据对象},另一个是\underline{对此数据对象进行操作的类型}
\myitem 存取控制机制包括\underline{自主存取控制}和\underline{强制存取控制}
\myitem 存储过程是存放在\underline{SQL server服务器上}的预定义并编译好的T-动态构建SQL语句
\myitem 游标是系统为用户开设的一个\underline{数据缓冲},存放SQL语句的执行结果

\myitem 什么是事务,事务有哪些特性?
\par 答:事务是DBMS的基本工作单位,它是用户定义的一组逻辑一致的程序序列。它是一个不可分割的工作单位,其中包含的所有操作,要么都执行,  要么都不执行。
事务具有4 个特性:原子性(Atomicity )、一致性(consistency )、隔离性( Isolation )和持续性(Durability )。这4 个特性也简称为ACID 特性。
原子性:事务是数据库的逻辑工作单位,事务中包括的诸操作要么都做,要么都不做。
一致性:事务执行的结果必须是使数据库从一个一致性状态变到另一个一致性状态。
隔离性:一个事务的执行不能被其他事务干扰。即一个事务内部的操作及使用的数据  对其他并发事务是隔离的,并发执行的各个事务之间不能互相干扰。持续性:持续性也称永久性(Perfnanence ) ,指一个事务一旦提交,它对数据库中数据的改变就应该是永久性的。接下来的其他操作或故障不应该对其执行结果有任何影响。
\myitem 事务中的提交和回滚是什么意思?
\par 答:事务中的提交(COMMIT)是提交事务的所有操作。具体说就是将事务中所有对数据库的更新写回到磁盘上的物理数 据库中去,事务正常结束。事务中的回滚(ROLLBACK)是数据库滚回到事务开始时 的状态。具体地说就是,在事务运行的过程中发生了某种故障,事务不能继续执行,系统将事务中对数据库的所有已完成的更新操作全部撤消,使数据库回滚到事务开始时的状态。
\myitem 为什么要设立日志文件?
\par 答:设立日志文件的目的,是为了记录对数据库中数据的每一次更新操作。从而DBMS可以根据日志文件进行事务故障的恢复和系统故障的恢复,并可结合后援 副本进行介质故障的恢复。  
\end{document}