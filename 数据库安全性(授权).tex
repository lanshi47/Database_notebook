 

\section{第四章:数据库安全性(授权)}
\subsection{不安全因素}
\begin{enumerate}
    \item ,,
    \item ,,,
\end{enumerate}
\subsection{数据库安全性控制}

\subsection{\color{red}\textbf{为什么授权?}}
\textbf{授权是指授予(GRANT)和收回(REVOKE),自主存取控制的方法,为了保护数据库防止不合法使用导致数据泄露更改或破坏}

\subsection{\color{red}\textbf{如何授权:授予GRANT}}
\begin{lstlisting}[language=SQL]
    GRANT 权限 ON 对象类型 对象名 TO 用户名 [WITH GRANT OPTION];
\end{lstlisting}
\begin{description}
    \item[权限] 这些是数据库访问的各种权限。例如 SELECT, INSERT, UPDATE, DELETE, CREATE, ALTER, DROP,以及所有权限的缩写 ALL PRIVILEGES 等等。多个权限之间用逗号分隔。
    \item[对象类型:] 这是数据库中可以授予权限的对象类型。常见的类型包括 TABLE, DATABASE, VIEW, FUNCTION, PROCEDURE 等。
    \item[对象名:] 这是具体的数据库对象的名称。比如,如果是表,则可以写表名。如果是数据库全局权限,则直接使用 *。
    \item[TO 用户名:] 指定接受权限的用户或角色。这里可以使用用户名(可能与数据库的用户系统有关)或角色名。如果需要授予权限给多个用户或角色,可以用逗号分隔。
    \item[WITH GRANT OPTION:] 这是可选的子句,它允许被授予权限的用户也能将这些权限再授予其他用户。如果没有这个选项,则用户只能使用这些权限,而不能再传递给其他人.
\end{description}
\paragraph*{示例}
\begin{enumerate}
    \item 给用户 user1 授予 employees 表的 SELECT 权限:
    \begin{lstlisting}[language=SQL]
        GRANT SELECT ON TABLE employees TO user1;
    \end{lstlisting}
    \item 授予 user1 对整个数据库 testDB 查看所有表(ALL TABLES IN SCHEMA)的SELECT权限:
    \begin{lstlisting}[language=SQL]
        GRANT SELECT ON ALL TABLES IN SCHEMA testDB TO user1;
    \end{lstlisting}
    \item 给用户 admin 授予对数据库的所有权限并允许其传递这些权限:
    \begin{lstlisting}[language=SQL]
        GRANT ALL PRIVILEGES ON DATABASE testDB TO admin WITH GRANT OPTION;
    \end{lstlisting}
\end{enumerate}
\textbf{注意:} SQL不允许循环授权(不能以下犯上)
\subsection{收回授权:收回 REVOKE}
\begin{lstlisting}[language=SQL]
    REVOKE 权限 ON 对象类型 对象名 FROM 用户名 [CASCADE][RESTRICT]
\end{lstlisting}
\paragraph*{权限} 
权限是用户在数据库中的操作许可,例如 SELECT, INSERT, UPDATE, DELETE 等。

\paragraph*{对象类型} 
这里指的是数据库中的对象,如 TABLE, VIEW, SEQUENCE, PROCEDURE 等。

\paragraph*{对象名} 
指定权限语句所作用的特定对象的名称。

\paragraph*{用户名} 
需要从中撤销权限的用户或角色名称。

\paragraph*{CASCADE}
如果指定了 CASCADE,当用户拥有权限再授权给其他用户时,撤销其权限也会撤销所有通过他间接获得的权限。

\paragraph*{RESTRICT} 
(可选的,非必要)如果用户已经将权限再授权给其他用户,将阻止该 REVOKE 操作。
使用 \lstinline|RESTRICT| 是为了确保不会不经意地删除其他用户对这些对象的访问权限。

\paragraph*{示例}
\begin{lstlisting}[language=SQL]
REVOKE SELECT ON TABLE employees FROM bob CASCADE;
\end{lstlisting}
这段语句会撤销用户 `bob` 的 `SELECT` 权限,并且由于使用了 `CASCADE`,任何通过 `bob` 获得的 `SELECT` 权限也会被撤销。

\paragraph*{注意}
\texttt{RESTRICT} 和 \texttt{CASCADE} 只能选择一个,不能同时使用。






