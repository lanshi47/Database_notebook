\section{数据库编程}
\subsection{嵌入式SQL与主语言之间的通信}

嵌入式SQL与主语言(如C、Java等)之间的通信主要通过以下几种方式进行:

\begin{enumerate}
    \item \textbf{SQL $\rightarrow$ 主语言}:
        \begin{itemize}
            \item \textbf{通信区(Communication Area, SQLCA)}:用于报告SQL语句执行的状态和错误信息。SQLCA是一个结构体或类,包含了SQL语句的执行结果、错误码、警告等信息。通过检查SQLCA,主语言程序可以获取SQL语句的执行情况并作出相应的响应。
        \end{itemize}
    
    \item \textbf{主语言 $\rightarrow$ SQL}:
        \begin{itemize}
            \item \textbf{主变量(Host Variables)}:主语言的变量可以直接用在嵌入式SQL语句中,将主语言的数据传递到数据库中。在SQL预处理阶段,这些变量会被相应地绑定到SQL语句中。
        \end{itemize}
    
    \item \textbf{查询结果 $\rightarrow$ 主语言}:
        \begin{itemize}
            \item \textbf{主变量和游标(Host Variables and Cursors)}:查询结果可以通过主变量直接返回,也可以使用游标来遍历返回的结果集。游标允许主语言程序逐行访问SQL查询(如SELECT语句)的结果数据。
        \end{itemize}
\end{enumerate}

\textbf{通信区(SQLCA)}:
\begin{itemize}
    \item SQLCA提供了一套结构化或对象化的方式来访问SQL语句执行后的状态和错误信息。具体的字段可能包括:
        \subitem SQLCODE(SQL代码):指示了执行SQL操作的状态,正值表示警告,负值表示错误。
        \subitem SQLERRM(错误信息):包含了描述错误或状况的文本信息。
\end{itemize}

\textbf{主变量}:
\begin{itemize}
    \item 在嵌入式SQL中,可以声明与外部语言兼容的变量,这些变量可以用作输入参数发送到SQL,也可以作为输出接收查询结果的容器。
\end{itemize}

\textbf{游标(Cursors)}:
\begin{itemize}
    \item 游标是一个控制结构,允许对查询结果集进行逐行或批量操作。它包括声明、打开、获取数据、关闭等几个步骤:
        \subitem \texttt{DECLARE}:声明游标。
        \subitem \texttt{OPEN}:打开游标执行查询。
        \subitem \texttt{FETCH}:从游标中获取一行或多行数据。
        \subitem \texttt{CLOSE}:关闭游标释放资源。
\end{itemize}

\subsection{相关示例代码}
下面是一个用C语言与嵌入式SQL(这里假设是使用了一种支持嵌入式SQL的编译器如Pro*C)的基本示例:

\begin{lstlisting}[language=C]
#include <stdio.h>
#include <string.h>

EXEC SQL INCLUDE SQLCA;

// 主变量声明
int id;
char name[20];

EXEC SQL BEGIN DECLARE SECTION;
int ID;
char NAME[20];
EXEC SQL END DECLARE SECTION;

int main() {
    EXEC SQL WHENEVER SQLERROR GOTO error;

    // 从用户获取ID
    printf("Enter ID: ");
    scanf("%d", &ID);
    
    // 查询语句,使用变量
    EXEC SQL SELECT name INTO :NAME FROM Employee WHERE id = :ID;
    
    // 打印结果
    printf("Employee Name foram ID %d is %s", ID, NAME);
    
    goto end;

error:
    printf("Error: %d - %s", sqlca.sqlcode, sqlca.sqlerrm.sqlerrmc);

end:
    return 0;
}
\end{lstlisting}